\documentclass[8pt,a4paper,twocolumn,table]{extarticle}
\usepackage[table]{xcolor}
\usepackage{mathtools}
\usepackage{multirow}
\usepackage{amsfonts}
\usepackage{amsthm}
\usepackage{amssymb}
\usepackage[margin=0.5cm]{geometry}
\usepackage{hyperref}
\setlength {\marginparwidth }{2cm}
\usepackage{todonotes}
\usepackage{xstring}
\usepackage{catchfile}

\setlength{\columnsep}{1cm}
\setlength{\columnseprule}{0.2pt}

\newcommand{\N}{\mathbb{N}}
\newcommand{\R}{\mathbb{R}}
\newcommand{\C}{\mathbb{C}}

\DeclareMathOperator{\arccot}{arccot}
\DeclareMathOperator{\acosh}{acosh}
\DeclareMathOperator{\asinh}{asinh}
\DeclareMathOperator{\atanh}{atanh}

% Math column
\newcolumntype{M}{>{$\displaystyle}c<{$}}
\newcolumntype{L}{>{$\displaystyle}l<{$}}
\newcolumntype{R}{>{$\displaystyle}r<{$}}

\newtheorem{innercustomgeneric}{\customgenericname}
\providecommand{\customgenericname}{}
\newcommand{\newcustomtheorem}[2]{%
  \newenvironment{#1}[1]
  {%
   \renewcommand\customgenericname{#2}%
   \renewcommand\theinnercustomgeneric{##1}%
   \innercustomgeneric
  }
  {\endinnercustomgeneric}
}

\newcustomtheorem{definition}{Definition}

\newcustomtheorem{lemma}{Lemma}

\newcustomtheorem{theorem}{Theorem}

\newcustomtheorem{corollary}{Corollary}

\newcommand{\seq}[1]{\left( #1_n \right)_{n \ge 1}}

% Read branchname and commit id from HEAD, to be added to the document
\CatchFileDef{\headfull}{.git/HEAD}{}
\StrGobbleRight{\headfull}{1}[\head]
\StrBehind[2]{\head}{/}[\branch]
\IfFileExists{.git/refs/heads/\branch}{%
    \CatchFileDef{\commit}{.git/refs/heads/\branch}{}
}{%
    \newcommand{\commit}{\dots~(in \emph{packed-refs})}
}

\title{Analysis 2 Recap}
\author{Axel Montini \\ \href{mailto:amontini@student.ethz.ch}{amontini@student.ethz.ch}}
\date{\today \hfill \texttt{\branch}, \texttt{\commit}}

\begin{document}
\maketitle

\section{Start}

\section{Limit Cheat Sheet}
\[ x \in \R,\quad a,b \in \R^+,\quad n \in \N \]
{\renewcommand{\arraystretch}{1.4}
\begin{tabular}{| M | M |}
    \hline
    \lim_{n \to \infty} a^n = +\infty\ \mbox{if}\ a > 1
                                                               &
    \lim_{n \to \infty} a^n = 0\ \mbox{if}\ 0 < a < 1            \\
    \lim_{n \to \infty} \sqrt[n]{n} = 1
                                                               &
    \lim_{n \to \infty} \sqrt[n]{a}\ \mbox{if}\ a > 0
    \\
    \lim_{n \to \infty} \left( 1 + \frac{x}{n} \right)^n = e^x &
    \\
    \hline
    \lim_{x \to 0} \frac{\sin x}{x} = 1
                                                               &
    \lim_{x \to 0} \frac{\tan x}{x} = 1
    \\
    \hline
    \lim_{x \to \infty} \frac{\ln^b(x)}{x^a} = 0
                                                               &
    \lim_{x \to 0} x^a \ln^b(x) = 0
    \\
    \lim_{x \to \infty} \frac{e^{ax}}{x^b} = +\infty
                                                               &
    \lim_{x \to \infty} \frac{x^b}{e^{ax}} = 0                   \\
    \hline
\end{tabular}}


\section{Derivative Cheat Sheet}

Properties
\begin{alignat*}{2}
     & (cf)' = c f'(x)     & (f \pm g)' = f'(x) \pm g'(x)                       \\
     & (fg)' = f'g + fg'   & (\frac{f}{g})' = \frac{f'g - fg'}{g^2}             \\
     & \frac{d}{dx}(c) = 0 & \frac{d}{dx}\left( g(f(x)) \right) = g'(f(x))f'(x)
\end{alignat*}
{\renewcommand{\arraystretch}{1.1}
\begin{tabular}{| R | L |}
    \hline
    \rowcolor{lightgray} f(x)      & f'(x)                                                    \\
    \hline
    x                              & 1                                                        \\
    x^n                            & nx^{n-1}                                                 \\
    e^x                            & e^x                                                      \\
    a^x                            & a^x \ln a                                                \\
    \sqrt{x}                       & \frac{1}{2\sqrt{x}},\ x \ne 0                            \\
    \frac{1}{x}                    & -\frac{1}{x^2}                                           \\
    |x|                            & x > 0 \implies 1,\ \mbox{or}\ x < 0 \implies -1, x \ne 0 \\
    \ln(x)                         & \frac{1}{x}                                              \\
    \log_a(x) = \frac{1}{x \ln(a)} &                                                          \\
    \hline % trig
    \sin(x)                        & \cos(x)                                                  \\
    \cos(x)                        & -\sin(x)                                                 \\
    \tan(x)                        & \frac{1}{\cos^2(x)} = 1 + \tan^2(x)                      \\
    \cot(x)                        & - \frac{1}{\sin^2(x)} = -1 - \cot^2(x)                   \\
    \arcsin x                      & \frac{1}{\sqrt{1 - x^2}}                                 \\
    \arccos x                      & -\frac{1}{\sqrt{1 - x^2}}                                \\
    \arctan x                      & \frac{1}{1 + x^2}                                        \\
    \arccot(x)                     & - \frac{1}{1 + x^2}                                      \\
    \hline % hyperb
    \sinh(x)                       & \cosh(x)                                                 \\
    \cosh(x)                       & \sinh(x)                                                 \\
    \tanh(x)                       & \frac{1}{\cosh^2(x)} = 1 - \tanh^2(x)                    \\
    \coth(x)                       & -\frac{1}{\sin^2(x)} = 1 - \coth^2(x)                    \\
    \asinh(x)                      & \frac{1}{\sqrt{x^2 + 1}}                                 \\
    \acosh(x)                      & \frac{1}{\sqrt{x^2 - 1}}                                 \\
    \atanh(x)                      & \frac{1}{1 - x^2}                                        \\
    \hline
\end{tabular}}

\section{Integral Cheat Sheet}

\[ \int f(x)dx = F(x) + C \]

\begin{tabular}{| l | L |}
    \hline
    Per parti                                                                & \int f'(x)g(x)dx = f(x)g(x) - \int f(x)g'(x)dx        \\
    \hline
    \multicolumn{1}{|p{2.2cm}|}{Per sostituzione immediata}                  & \int g(f(x))f'(x)dx = G(f(x)) + C                     \\
    \hline
    \multicolumn{1}{|p{2.2cm}|}{Per sostituzione (cambiamento di variabile)} & \int g(x)dx = \int g(f(t)) f'(t) dt                   \\
                                                                             & \mbox{with}\ x = f(t)                                 \\
    \hline
    \multicolumn{1}{|p{2.2cm}|}{Integrale logaritmico}                       & \int \frac{f'(x)}{f(x)} dx = \ln \Big| f(x) \Big| + C \\
    \hline
\end{tabular}

{\renewcommand{\arraystretch}{1}
\begin{tabular}{| M | M |}
    \hline
    \rowcolor{lightgray} f(x)  & F(X) (without + C)                                                              \\
    \hline
    a                          & ax                                                                              \\
    x^n                        & \frac{x^{n + 1}}{n + 1}                                                         \\
    \frac{1}{x}                & \ln |x|                                                                         \\
    \sqrt{x}                   & \frac{2}{3}x\sqrt{x}                                                            \\
    \frac{1}{\sqrt{x}}         & 2 \sqrt{x}                                                                      \\
    \frac{1}{(x - a)(x - b)}   & \frac{1}{a - b}\ln \left| \frac{x - a}{x - b}\right|                            \\
    \frac{ax + b}{cx + d}      & \frac{ax}{c} - \frac{ad - bc}{c^2}\ln |cx + d|                                  \\
    \frac{1}{x^2 + a^2}        & \frac{1}{a} \arctan\left( \frac{x}{a} \right)                                   \\
    \frac{1}{x^2 - a^2}        & \frac{1}{2a} \ln \left| \frac{x - a}{x + a}\right|                              \\
    \hline
    e^x                        & e^x                                                                             \\
    \ln(x)                     & x(\ln(x) - 1)                                                                   \\
    a^x                        & \frac{a^x}{\ln(a)}                                                              \\
    \log_a(x)                  & x(\log_a(x) - \log_a(e))                                                        \\
    xe^{ax}                    & \frac{1}{a^2}(ax - 1)e^{ax}                                                     \\
    x\ln(ax)                   & \frac{x^2}{4}(2\ln(ax) - 1)                                                     \\
    \hline
    \sin(x)                    & -\cos(x)                                                                        \\
    \arcsin(x)                 & x\arcsin(x) + \sqrt{1 - x^2}                                                    \\
    \cos(x)                    & \sin(x)                                                                         \\
    \arccos(x)                 & x\arccos(x) - \sqrt{1 - x^2}                                                    \\
    \tan(x)                    & -\ln|\cos(x)|                                                                   \\
    \arctan(x)                 & x\arctan(x) - \frac{1}{2}\ln(1+x^2)                                             \\
    \cot(x)                    & \ln|\sin(x)|                                                                    \\
    \arccot(x)                 & x\arccot(x) + \frac{1}{2}\ln(1+x^2)                                             \\
    \hline
    \sin^2(x)                  & \frac{1}{2}(x - \sin(x)\cos(x))                                                 \\
    \cos^2(x)                  & \frac{1}{2}(x + \sin(x)\cos(x))                                                 \\
    \tan^2(x)                  & \tan(x) - x                                                                     \\
    \hline
    \sqrt{x^2 + a}             & \frac{1}{2}x\sqrt{x^2 + a} + \frac{a}{2}\ln \left| x + \sqrt{x^2} + a \right|   \\
    \frac{1}{\sqrt{x^2 + a}}   & \ln \left| x + \sqrt{x^2 + a}\right|                                            \\
    \sqrt{r^2 - x^2}           & \frac{1}{2}x\sqrt{r^2 - x^2} + \frac{r^2}{2} \arcsin \left( \frac{x}{r} \right) \\
    \frac{1}{\sqrt{r^2 - x^2}} & \arcsin\left( \frac{x}{r} \right)                                               \\
    \hline
\end{tabular}}

\todo[inline]{More?}

\end{document}