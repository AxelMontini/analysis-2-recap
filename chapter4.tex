\subsection{Line integrals}
\begin{definition}{4.1.1}
    Uses scalar product in $\R^n$.

    \begin{enumerate}
        \item Let $I = [a,b]$ be a closed and bounded interval in $\R$. Let $f(t) = (f_1(t), f_2(t), ..., f_n(t))$
              be continuous ($f_i$ is continuous). Then we define
              \[ \int_a^b f(t) dt = \left( \int_a^b f_1(t) dt, ..., \int_a^b f_n(t) dt \right) \]
        \item A \textbf{parametrized curve} in $\R^n$ is a continuous map $\gamma: [a,b] \mapsto \R^n$ that is piecewise $C^1$, i.e,
              there's $k \ge 1$ and a partition
              \[ a = t_0 < t_1 < ... < t_{k-1} < t_k = b \]
              such that the restriction of $f$ to $]t_{j-1}, t_j[$ is $C^1$ for $1 \le j \le k$. Then we say that $\gamma$ is a parametrized
              curve, or \textit{path}x, between $\gamma(a)$ and $\gamma(b)$.
        \item Let $gamma: [a,b] \mapsto \R^n$ be a parametrized curve. Let $X \subset \R^n$ be a subset containing the image of $\gamma$.
              Let $g: X \mapsto \R^n$ be a continuous function. Then the integral
              \[ \int_a^b g(\gamma(t))\gamma'(t) dt\ \in \R \]
              is called the \textbf{line integral} of $g$ along $\gamma$. Denoted
              \[ \int_\gamma g(s)\cdot ds \quad \mbox{or} \quad \int_\gamma g(s) \cdot d\vec{s}\]
    \end{enumerate}
\end{definition}

When working with line integrals, we say that $f: X \mapsto \R^n$ is a \textbf{vector field}.

\begin{proposition}{idk}
    This integral  of continuous functions $I \mapsto \R^n$ (one variable) satisfies
    \[ \int_a^b (f(t) + g(t))dt = \int_a^b f(t) dt + \int_a^b g(t)dt \]
    and
    \[ \int_a^b f(t)dt = - \int_b^a f(t)dt \]
\end{proposition}

\begin{definition}{4.1.4}
    Let $gamma: [a,b] \mapsto \R^n$ be a parametrized curve. An \textbf{oriented reparametrization} of $\gamma$ is
    a parametrized curve $\sigma: [c,d] \mapsto \R^n$ such that $\sigma = \gamma \circ \varphi$, differentiable on $]a,b[$, strictly
    increasing and satisfies $\varphi(a) = c, \varphi(b) = d$, where $\varphi: [c,d] \mapsto [a,b]$ is a continuous map.
\end{definition}

\begin{proposition}{4.1.5}
    Let $\gamma$ be a parametrized curve in $\R^n$, $\sigma$ an oriented reparametrization of $\gamma$.
    Let $X$ be a set containing the image of $\gamma$ (or, equivalently, the image of $\sigma$),
    and $f: X \mapsto \R^n$ a continuous function.

    Then the line integrals are the same:
    \[ \int_\gamma f(s) \cdot d \vec{s} = \int_\sigma f(s) \cdot d \vec{s} \]
\end{proposition}

\begin{definition}{4.1.8}
    Let $X \subset \R^n$ and $f: X \mapsto \R^n$ a continuous vector field.

    If for any $x_1, x_2 \in X$ the line integral is independent of the choice of $\gamma$ in $X$ from $x_1$ to $x_2$,
    then we say that the vector field is \textbf{conservative}.
\end{definition}

\begin{remark}{4.1.9}
    Equivalently, $f$ is conservative iff
    \[ \int_\gamma f(s) \cdot d\vec{s} = 0 \]
    for any \textbf{closed} parametrized curve $\gamma$ in $X$.
    A curve is closed if $\gamma(a) = \gamma(b)$.
\end{remark}

\begin{theorem}{Hidden in the page (gratient vector conservative)}
    If $X$ is open, then any vector field of the form $f = \nabla g$, where $g$ is of class $C^1$ on $X$, is conservative.
\end{theorem}

\begin{theorem}{4.1.10}
    Let $X$ be open and $f$ a conservative vector field.
    Then there exist a $C^1$ function $g$ on $X$ such that $f = \nabla g$.

    If any two points on $X$ can be joined by a parametrized curve, then $g$ is unique up to addition of a constant:
    if $\nabla g_1 = f$, then $g - g_1$ is constant on $X$.
\end{theorem}

\begin{remark}{4.1.11}
    To say that Any two points of $X$ can be joined by a parametrized curve means that,
    for all $x,y \in X$, there exist a p.c. $\gamma: [a,b] \mapsto X$ such that $\gamma(a) = x, \gamma(b) = y$.
    When this is true, we say that $X$ is \textbf{path-connected}.

    (it is true whenever $X$ is \textbf{convex})

    If $f$ is a conservative vector field on $X$, then a function $g$ such that $\nabla g = f$ is called a \textbf{potential} for $f$.
    Note that $g$ is not unique and can differ of at least a constant.
\end{remark}

\begin{proposition}{4.1.13}
    Let $X \subset \R^n$ be an open set, $f: X \mapsto \R^n$ a vector field of class $C^1$. Write
    \[ f(x) = (f_1(x), ..., f_n(x)) \]
    If $f$ is conservative, then we have, for any integers with $1 \le i \ne j \le n$:
    \[ \frac{\partial f_i}{\partial x_j} = \frac{\partial f_j}{\partial x_i} \]
\end{proposition}

\begin{example}{4.1.14}
    Consider $f(x,y,z) = (y^2, xz, 1)$.
    Clearly, $\partial_y (y^2) = 2y \ne z = \partial_x (xz)$
    Then $f$ can't be conservative.
\end{example}

\begin{definition}{4.1.15}
    A subset $X \subset \R^n$ is \textbf{star shaped} if there exists $x_0 \in X$ so
    that for all $x \in X$, the line connecting the two is contained in $X$.

    Then $X$ is star-shaped around $x_0$.
\end{definition}

\begin{theorem}{4.1.17}
    Let $X \subset \R^n$ be star shaped and $f$ be a $C^1$ vector field such that on $X$, for any $1 \le i \ne j \le n$:
    \begin{equation}
        \label{eq:frac-partial-conservative}
        \frac{\partial f_i}{\partial x_j} = \frac{\partial f_j}{\partial x_i}
    \end{equation}

    Then $f$ is \textbf{conservative}
\end{theorem}

\begin{definition}{4.1.20}
    Let $X \subset \R^3$ be an open set and $f: X \mapsto \R^3$ a $C^1$ vector field.
    Then the curl of $f$, $curl(f)$, is the continuous vector field on $X$
    \[ curl(f) = \begin{pmatrix}
            \partial_y f_3 - \partial_z f_2 \\
            \partial_z f_1 - \partial_x f_3 \\
            \partial_x f_2 - \partial_y f_1
        \end{pmatrix} \]
    Where $f(x,y,z) = (f_1(x,y,z), f_2(x,y,z), f_3(x,y,z))$

    If $curl(f) = 0$, \textbf{then condition \eqref{eq:frac-partial-conservative} holds!}
\end{definition}

\begin{remark}{4.1.21 (remember the definition with determinant)}
    With $(e_1, e_2, e_3)$ being the canonical basis of $\R^3$ and expanding
    with $\partial_x \cdot f_i = f_i \partial_x = \partial_x f_i$:
    \[
        curl(f) = \begin{vmatrix}
            e_1        & e_2        & e_3        \\
            \partial_x & \partial_y & \partial_z \\
            f_1        & f_2        & f_3
        \end{vmatrix}
    \]
\end{remark}

\subsection{The Riemann integral in $\R^n$}

For any bounded closed $X \subset \R^n$ and continuous function $f: X \mapsto \R$,
one can define \textit{the integral of $f$ over $X$}, denoted
\[ \int_X f(x) dx \quad \in \R\]

The integral satisfies the properties:
\begin{enumerate}
    \item \textbf{(Compatibility)} if $n = 1$ and $X = [a,b]$ is an interval ($a \le b$),
          then the integral of $f$ over $X$ is the Riemann integral of $f$:
          \[ \int_{[a,b]} f(x)dx = \int_a^b f(x)dx \]
    \item \textbf{(Linearity)} if $f,g$ are continuous on $X$ and $a,b \in \R$, then
          \[ \int_X (a f_1(x) + b f_2(x))dx = a \int_X f_1(x)dx + b \int_X f_2(x)dx \]
    \item \textbf{(Positivity)} if $f \le g$, then
          \[ \int_X f(x) dx \le \int_X g(x)dx \]
          and especially, if $f \ge 0$, then
          \[ \int_X f(x)dx \ge 0 \]
          Moreover, if $Y \subset X$ is compact and $f \ge 0$, then
          \[ \int_Y f(x)dx \le \int_X f(x)dx \]
    \item \textbf{(Upper bound and triangle inequality)} Since $-|f| \le f \le |f|$,
          we have
          \[ \Big| \int_X f(x)dx \Big| \le \int_X \Big| f(x) \Big| dx\]
          and since $|f + g| \le |f| + |g|$
          \[ \Big| \int_X (f(x) + g(x)) dx \Big| \le \int_X \Big| f(x)\Big|dx + \int_X \Big| g(x) \Big| dx\]
    \item \textbf{(Volume)} if $f = 1$, then the integral of $f$ is the \textit{"volume"} in $\R^n$ of the set $X$. If
          $f \ge 0$ in general, the integral of $f$ is the volume of the set
          \[ \left\{ (x,y) \in X \times \R \mid 0 \le y \le f(x) \right\} \subset \R^{n+1} \]

          In particular, if $X$ is a bounded "rectangle", say
          \[ X = [a_1,b_1] \times ... \times [a_n, b_n] \subset \R^n \]
          and $f = 1$, then
          \[ \int_X dx = (b_n - a_n) \cdot \dots \cdot (b_1 - a_1) \]
          We write $Vol(X)$ or $Vol_n(X)$ for the volume of $X$.

    \item \textbf{(Multiple Integral, or Fubini's Theorem)} If $n_1, n_2 \ge 1$ are integers such that
          $n = n_1 + n_2$, then for $x_1 \in \R^{n_1}$, let
          \[ Y_{x_1} = \left\{ x_2 \in \R^{n_2} \mid (x_1, x_2) \in X \right\} \subset \R^{n_2} \]
          Let $X_1$ be the set of $x_1$ such that $Y_{x_1}$ is not empty. Then $X_1$ is compact in $\R^{n_1}$
          and $Y_{x_1}$ is compact in $\R^{n^2}$ for all $x_1 \in X_1$. If the function
          \[ g(x_1) = \int_{Y_{x_1}} f(x_1, x_2)dx_2 \]
          on $X_1$ is continuous, then
          \[ \int_X f(x_1, x_2)dx = \int_{X_1} g(x_1)dx_1 = \int_{X_1} \left( \int_{Y_{x_1}} f(x_1, x_2)dx_2 \right) dx_1\]
          Similarly, exchanging the role of $x_1$ and $x_2$, we have
          \[ \int_X f(x_1, x_2)dx = \int_{X_2} \left( \int_{Z_{x_2}} f(x_1, x_2)dx_1 \right) dx_2 \]
          Where $Z_{x_2} = \left\{ x_1 \mid (x_1, x_2) \in X \right\}$, if the integral over $x_1$ is a continuous function.

    \item \textbf{(Domain additivity)} if $X_1$ and $X_2$ are compact subsets of $\R^n$ and $f$ is continuous on $X_1 \cup X_2$, then
          \[ \int_{X_1 \cup X_2} f(x)dx + \int_{X_1 \cap X_2} f(x)dx = \int_{X_1} f(x) dx + \int_{X_2} f(x)dx \]

          Notice that $X_1 \cap X_2$ is also compact, so all integrals exist.
          If $X_1 \cap X_2$ is empty, then the integral over it is equal to $0$,
          shortening the formula to a more convenient form.
          This is also true if the intersection is \textbf{negligible} (Def \ref{def:4.2.3}).
\end{enumerate}

\begin{definition}{4.2.3}
    \label{def:4.2.3}
    \begin{enumerate}
        \item Let $1 \le m \le n$. A \textbf{parametrized m-set} in $\R^n$ is a contiuous map
              \[ f: [a_1, b_1] \times ... \times [a_m, b_m] \mapsto \R^n \]
              which is $C^1$ on
              \[ ]a_1, b_1[ \times ... \times ]a_m, b_m[ \]
        \item A subset $B \subset \R^n$ is \textbf{negligible} if there exist an integer $k \ge 0$ and parametrized
              $m_i$-sets $f_i: X_i \mapsto \R^n$ with $1 \le n \le k$ and $m_i < n$ such that
              \[ X \subset f_1(X_1) \cup ... \cup f_k(X_k) \]
    \end{enumerate}
\end{definition}

\begin{example}{4.2.4}
    Any subset of the real axis $\R \times \{0\}$ is negligible in $\R^2$.

    More generally, if $H \subset \R^n$ is an affine subspace of dimension $m < n$, then any subset of $\R^n$ that is contained in $H$ is negligible.
\end{example}

\begin{proposition}{4.2.5}
    Let $X \subset \R^n$ be a compact set. Assume that $X$ is negligible.
    Then for any continuous function on $X$ we have
    \[ \int_X f(x)dx = 0 \]
\end{proposition}

% //TODO: Jordan-measurable subsets and, in general, REMARK 4.2.7?

\subsection{Improper integrals}
Some basic definitions on $\R^2$.

Let $I \subset \R$ be a bounded interval, $J = [a, +\infty[$ for some $a \in \R$.
Let $f$ be a continuous function on $X = J \times I$.
We say that $f$ is \textbf{Riemann-integrable} on $X$ if the following limit exists
\[ \lim_{x \to +\infty} \int_{[a,x] \times I} f(x,y) dx dy = \lim_{x \to +\infty} \int_I \left( \int_a^x f(x,y)dx \right)dy \]
The equality is a case of Fubini's Theorem.
We denote this limit with
\[ \int_{J \times I} f(x,y) dx dy \]


Similarly, let $f$ be continuous on $\R^2$. Assume that $f \ge 0$. We say that $f$ is Riemann-integrable on
$\R^2$ if the following limit exists
\[ \lim_{R \to +\infty} \int_{[-R, R]^2} f(x,y)dxdy \]
This limit is called the \textbf{integral of $f$ over $\R^2$} and denoted
\[ \int_{\R^2} f(x, y)dxdy \]
Fubini formula for this:
\[ \int_{\R^2} f(x,y)dxdy = \int_{-\infty}^{+\infty} \left( \int_{-\infty}^{+\infty} f(x,y)dx \right) dy \]

\begin{remark}{4.3.1}
    In all these cases we also often say that "the integral converges"
    to indicate that a function is Riemann-integrable on an unbounded set.

    If $|f| \le g$ and the integral of $g$ is Riemann-integrable on an unbounded set, then $f$ also
    does.
\end{remark}

\subsection{The change of variable formula}

Analogue of the one for one-variable calculus
\[ \left( \int f(g(x))g'(x)dx = \int f(y) dy \right) \]

Let $\bar{X}, \bar{Y} \subset \R^n$ be compact subsets. Let $\varphi: \bar{X} \mapsto \bar{Y}$ be a continuous map.

We assume that we can write $\bar{X} = X \cup B$ and $\bar{Y} = Y \cup C$ where
\begin{itemize}
    \item the sets $X,Y$ are open.
    \item The sets $B,C$ are negligible (Def \ref:{def:4.2.3})
    \item the restriction of $\varphi$ to the open set $X$ is a $C^1$ bijective map from $X$ to $Y$.
\end{itemize}

Then $J_\varphi(x)$ is invertible at all $x \in X$.
We assume that we can find a continuous function on $\bar{X}$ that restricts to $\det(J_\varphi(x))$ on $X$ (we have a formula for the Jacobian,
so this is obvious in most cases). Abuse notation and write it even if $x \in B$.

\begin{remark}{4.4.1}
    There is no assumption concerning the image of $B$.

    Sometimes $\varphi$ is the restriction of a $C^1$ map $\R^n \mapsto \R^n$, in which case the last issue doesn't require any argument.
\end{remark}

\begin{theorem}{4.4.2 (Change of variable formula)}
    \label{thm:4.4.2}
    In the situation described above, for any continuous function $f$ on $\bar{Y}$, we have
    \[ \int_{\bar{X}} f(\varphi(x))|\det(J_\varphi(x))|dx = \int_{\bar Y} f(y)dy \]
\end{theorem}

To remember, when $y = \varphi(x)$, then $dy = |\det(J_\varphi(x))|dx$.

Special cases:
\begin{enumerate}
    \item When $\varphi(x) = x + x_0$ (translation): $\varphi$ is affine-linear and $J_\varphi(x) = 1_n$ (identity matrix). The
          change of variable formula becomes, for any compact subset $\bar X$ and any continuous function $f$ on $x_0 + \bar X$:
          \[ \int_{\bar X} f(x + x_0)dx = \int_{x_0 + \bar X} f(x) dx \]

    \item When $\varphi$ is a restriction of a bijective linear map, namely $\varphi(x) = Ax$, where $A$ is an invertible matrix of size $n$.
          Then $J_\varphi(x) = A$ for all $x \in \R^n$ with constant determinant $\det(A)$.
          Let $\bar X = X \cup B$ be compact as above and $\bar Y = \varphi(\bar X)$. Then $\varphi(\bar X) = \varphi(X) \cup \varphi(B)$.
          The change of variable formula becomes (for any continuous $f$ on $\bar Y$)
          \[ \int_{\bar X} f(\varphi(x))dx = \frac{1}{|\det(A)|} \int_{\bar Y} f(y)dy \]
\end{enumerate}

\begin{example}{(Standard examples)}
    \begin{enumerate}
        \item \textbf{Polar coordinates} $(r, \theta)$ are useful for integrating over a disc in $\R^2$ centered
              at $0$, or more generally over a disk sector $\delta = \delta(a,b,R)$ defined
              \[ 0 \le r \le R, \quad -\pi < a \le \theta \le b < \pi\]
              We compute the Jacobian det. and obtain
              \[ \int_\delta f(x,y) dx dy = \int_0^R \int_a^b f(r \cos \theta, r \sin \theta)r\ dr\ d\theta \]
        \item \textbf{Spherical Coordinates} $(r, \theta, \varphi)$ in $\R^3$, integrate on balls centered at $0$.
              Jacobian determinant is $-r^2 \sin(\varphi)$. To integrate a function $f$ over a ball of radius $R$ we use
              \[ \int_B f(x,y,z) dx dy dz = \int_0^R \int_0^{2\pi} \int_0^\pi f(\bar x, \bar y, \bar z) r^2 \sin(\varphi)\ dr\ d\theta\ d\varphi \]
              With $\bar x = r \cos(\theta) \sin(\varphi)$, $\bar y = r \sin(\theta) \sin(\varphi)$, $\bar z = r \cos(\varphi)$.
    \end{enumerate}
\end{example}

\subsection{Geometric applications of integrals}

Welp, applications that can actually be useful

\begin{enumerate}
    \item \textbf{(Center of mass)} Let $X$ be a compact subset of $\R^n$ of positive volume. The \textit{center of mass} (or \textit{barycenter}) of
    $X$ is the point $\bar x \in \R^n$ such that $\bar x = (\bar x_1, ..., \bar x_n)$ with
    \[ \bar x_1 = \frac{1}{Vol{X}} \int_X x_i dx \]
    Intuitively, $x_i$ is the average over $X$ of the $i$-th coordinate and $\bar x$ is the point where $X$ is "perfectly balanced" (could be outside of $X$!).

    \item \textbf{(Surface area)} Consider a function $f: [a,b] \times [c,d] \mapsto \R$ which is $C^1$ on the open interval. Let
    \[ \Gamma = \left\{ (x,y,z) \in \R^3 \mid (x,y) \in [a,b] \times [c,d],\ z = f(x,y) \right\} \subset \R^3 \]
    be the graph of $f$. Intuitively, this is a surface and it has area
    \[ \int_a^b \int_c^d \sqrt{1 + (\partial_x f(x,y))^2 + (\partial_y f(x,y))^2}dxdy \]

    Analogue for a function $f: [a,b] \mapsto \R$ (\textbf{length}):
    \[ \int_a^b \sqrt{1 + f'(x)^2}dx \]
\end{enumerate}


\subsection{The Green formula}

\begin{definition}{4.6.1}
    A \textbf{simple closed parametrized curve} $\gamma: [a,b] \mapsto \R^2$ is a closed ($\gamma(a) = \gamma(b)$) parametrized
    curve such that $\gamma(t) \ne \gamma(s)$ unless $t = s$ or they are $a,b$, and such
    that $\gamma'(t) \ne 0$ for $a < t < b$ (if $\gamma$ is only piecewise $C^1$, this condition only applies where $\gamma'(t)$ exists).
\end{definition}

\begin{theorem}{4.6.3 (Green's formula)}
    Let $X \subset \R^2$ be a compact set with a boundary $\partial X$ that is the union of finitely many simple closed
    parametrized curves $\gamma_1, ..., \gamma_k$. Assume that
    \[ \gamma_i: [a_i, b_i] \mapsto \R^2 \]
    has the property that $X$ lies always "to the left" of the tangent vector $\gamma'(t)$ based at $\gamma_i(t)$.

    Let $f = (f_1, f_2)$ be a vector field of class $C^1$ defined on some open set containing X. Then we have
    \[ \int_X \left( \frac{\partial f_2}{\partial x} - \frac{\partial f_1}{\partial y} \right)dxdy = \sum_{i=1}^k \int_{\gamma_i} f \cdot d\vec{s} \]

    If the orientation is not met, the curve can be "reversed", e.g., replaced with $\tilde \gamma = \gamma(1 - t)$, which
    reverses the orientation of the tangent vector.
\end{theorem}

\begin{corollary}{4.6.5}
    Let $X \subset \R^2$ compact set with boundary $\partial X$ that is the union of finitely many s.c.p.c $\gamma_1, ..., \gamma_k$.
    Assume that 
    \[ \gamma_i = (\gamma_{i,1}, \gamma_{i,2}): [a_i, b_i] \mapsto \R^2 \]
    has the property that $X$ always lies "left" of the tangent vector. Then we have
    \[ Vol(X) = \sum_{i=1}^k \int_{\gamma_i} x \cdot d \vec{s} = \sum_{i=1}^k \int_{a_i}^{b_i} \gamma_{i,1}(t) \gamma_{i,2}'(t)dt \]
\end{corollary}

\subsection{The Gauss-Ostrogradski formula}

Analogue of the Green formula in $\R^3$.

\begin{definition}{4.7.1}
    A parametrized surface $\Sigma: [a,b] \times [c,d] \mapsto \R^3$ is a 2-set in $\R^3$ such that the rank
    of the J. matrix is $2$ at all $(s,t) \in ]a,b[ \times ]c,d[$

    Note that 2 is the max rank (there are two variables).
\end{definition}

\begin{definition}{4.7.3 (vector product)}
    Let $x,y$ be two linearly independent vectors in $\R^3$.
    The vector product (or cross product) $z = x \times y$ is the unique vector such that $(x,y,z)$ is a basis of $\R^3$ ($z$ perpendicular
    to the plane generated by $x,y$, also pairwise lin. indep.) with $\det(x,y,z) \ge 0$ and
    \[ \Vert z \Vert = \Vert x \Vert \cdot \Vert y \Vert \cdot \sin(\theta) \]
    Where $\theta = \angle(x,y)$

    If $x,y$ are not lin. indep., then we define $x \times y = 0$, the zero vector.
\end{definition}

Ez formula for \textit{canonical base only (remember the oral exam back in the day!)}:
\[
    x \times y = \begin{pmatrix}
        x_2y_3 - x_3y_2\\
        x_3y_1 - x_1 y_3\\
        x_1 y_2 - x_2 y_1
\end{pmatrix}
=
\det \begin{vmatrix}
    e_1 & e_2 & e_3\\
    x_1 & x_2 & x_3 \\
    y_1 & y_2 & y_3
\end{vmatrix}
\]

Also $y \times x = -x \times y$.

\begin{theorem}{4.7.6 (Gauss-Ostrogradski formula)}
    Let $X \subset \R^3$ be a compact set with a boundary $\partial X$ that is a parametrized surface $\Sigma: [a,b] \times [c,d] \mapsto \R^3$.

    Assume that $\Sigma$ is injective in the open interval, and that the normal vector of $\Sigma$ points away from the surface at all points.

    Let $\vec u = \frac{\vec n}{\Vert \vec n \Vert }$ be the unit exterior normal vector.

    Let $f = (f_1, f_2, f_3)$ be a $C^1$ vector field defined on some open set containing $X$. Then we have
    \[ \int_X div(f) dx dy dz = \int_\Sigma (f \cdot \vec u) d\sigma \]

    Clarify: $\div(f)$ is the divergence of the vector field $f$, $div(f) = \partial_x f + \partial_y f + \partial_z f$.
\end{theorem}