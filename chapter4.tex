\subsection{Line integrals}
\begin{definition}{4.1.1}
    Uses scalar product in $\R^n$.

    \begin{enumerate}
        \item Let $I = [a,b]$ be a closed and bounded interval in $\R$. Let $f(t) = (f_1(t), f_2(t), ..., f_n(t))$
              be continuous ($f_i$ is continuous). Then we define
              \[ \int_a^b f(t) dt = \left( \int_a^b f_1(t) dt, ..., \int_a^b f_n(t) dt \right) \]
        \item A \textbf{parametrized curve} in $\R^n$ is a continuous map $\gamma: [a,b] \mapsto \R^n$ that is piecewise $C^1$, i.e,
              there's $k \ge 1$ and a partition
              \[ a = t_0 < t_1 < ... < t_{k-1} < t_k = b \]
              such that the restriction of $f$ to $]t_{j-1}, t_j[$ is $C^1$ for $1 \le j \le k$. Then we say that $\gamma$ is a parametrized
              curve, or \textit{path}x, between $\gamma(a)$ and $\gamma(b)$.
        \item Let $gamma: [a,b] \mapsto \R^n$ be a parametrized curve. Let $X \subset \R^n$ be a subset containing the image of $\gamma$.
              Let $g: X \mapsto \R^n$ be a continuous function. Then the integral
              \[ \int_a^b g(\gamma(t))\gamma'(t) dt\ \in \R \]
              is called the \textbf{line integral} of $g$ along $\gamma$. Denoted
              \[ \int_\gamma g(s)\cdot ds \quad \mbox{or} \quad \int_\gamma g(s) \cdot d\vec{s}\]
    \end{enumerate}
\end{definition}

When working with line integrals, we say that $f: X \mapsto \R^n$ is a \textbf{vector field}.

\begin{proposition}{idk}
    This integral  of continuous functions $I \mapsto \R^n$ (one variable) satisfies
    \[ \int_a^b (f(t) + g(t))dt = \int_a^b f(t) dt + \int_a^b g(t)dt \]
    and
    \[ \int_a^b f(t)dt = - \int_b^a f(t)dt \]
\end{proposition}

\begin{definition}{4.1.4}
    Let $gamma: [a,b] \mapsto \R^n$ be a parametrized curve. An \textbf{oriented reparametrization} of $\gamma$ is
    a parametrized curve $\sigma: [c,d] \mapsto \R^n$ such that $\sigma = \gamma \circ \varphi$, differentiable on $]a,b[$, strictly
    increasing and satisfies $\varphi(a) = c, \varphi(b) = d$, where $\varphi: [c,d] \mapsto [a,b]$ is a continuous map.
\end{definition}

\begin{proposition}{4.1.5}
    Let $\gamma$ be a parametrized curve in $\R^n$, $\sigma$ an oriented reparametrization of $\gamma$.
    Let $X$ be a set containing the image of $\gamma$ (or, equivalently, the image of $\sigma$),
    and $f: X \mapsto \R^n$ a continuous function.

    Then the line integrals are the same:
    \[ \int_\gamma f(s) \cdot d \vec{s} = \int_\sigma f(s) \cdot d \vec{s} \]
\end{proposition}

\begin{definition}{4.1.8}
    Let $X \subset \R^n$ and $f: X \mapsto \R^n$ a continuous vector field.

    If for any $x_1, x_2 \in X$ the line integral is independent of the choice of $\gamma$ in $X$ from $x_1$ to $x_2$,
    then we say that the vector field is \textbf{conservative}.
\end{definition}

\begin{remark}{4.1.9}
    Equivalently, $f$ is conservative iff
    \[ \int_\gamma f(s) \cdot d\vec{s} = 0 \]
    for any \textbf{closed} parametrized curve $\gamma$ in $X$.
    A curve is closed if $\gamma(a) = \gamma(b)$.
\end{remark}

\begin{theorem}{Hidden in the page (gratient vector conservative)}
    If $X$ is open, then any vector field of the form $f = \nabla g$, where $g$ is of class $C^1$ on $X$, is conservative.
\end{theorem}

\begin{theorem}{4.1.10}
    Let $X$ be open and $f$ a conservative vector field.
    Then there exist a $C^1$ function $g$ on $X$ such that $f = \nabla g$.

    If any two points on $X$ can be joined by a parametrized curve, then $g$ is unique up to addition of a constant:
    if $\nabla g_1 = f$, then $g - g_1$ is constant on $X$.
\end{theorem}

\begin{remark}{4.1.11}
    
\end{remark}